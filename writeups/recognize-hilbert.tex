\documentclass{article}
\title{Recognizing Hilbert Class Polynomials}
\author{Hao Chen}

\usepackage{macros}


\begin{document}
\maketitle

\section{The problem}

The problem is to develop an fast algorithm that: takes as input a polynomial $f(x) \in \bZ[x]$, returns a negative discriminant $D < 0$ if $f(x) = H_D(x)$, and returns {\it Null} if $f$ is not a Hilbert class polynomial.

\section{The analysis}

First, obviously, we should check if $f$ is irreducible. \\

Next, we consider the field $L_f = \bQ[x]/(f(x))$. Assume $f = H_D$, we set $K = \bQ(\sqrt{D})$ and $H = K[x]/(f(x))$ be the corresponding ring class field of $K$. Consider the following diagram:

\begin{cd}
& H \arrow[dl] \arrow[dr] & \\
K \arrow[dr] & & L_f \arrow[dl] \\
& \bQ &
\end{cd}

\subsection{Case 1: $L_f$ is Galois over $\bQ$}
We want to say that usually $H/\bQ$ is the Galois closure of $L_f/\bQ$. This is equivalent to saying that $L_f/\bQ$ is not Galois. \\
Suppose $L_f$ is Galois. Then the group $\Delta = \Gal(H/L_f)$ is a normal subgroup of order two in $G = \Gal(H/\bQ)$. Note that $\Delta$ is generated by a lift of the complex conjugation $\sigma$ on $K$. By Cox,  we know that the conjugation action of $\sigma$ on $\Gal(H/K)$ is
\[
    \sigma \tau \sigma^{-1}  = \tau^{-1}.
\]
However, $\Delta$ is a normal subgroup of $G$. So $\sigma \tau = \tau \sigma$. Hence we must have that $\Gal(H/K)$ has exponent 2,
which means that the class group $C(D)$ is elementary 2-abelian,
which is equivalent to $D$ being a "convenient number" in the sense of Euler.

Note that in this case $h_D$ is necessarily a power of 2. Hence if $\deg f$ is not a power of 2, this can not happen.


\subsection{Case 2: $L_f/\bQ$ is not Galois}

In this case, since $H$ is the Galois closure of $L_f$ over $\bQ$, we know from Ralph's class that the set of ramified primes of the two extensions are the same. At the same time, we know that
$S(H/\bQ)$ is contained in the primes dividing $D$, and it contains the primes dividing $d_K$.

(Question: is it true that $S(H/\bQ)$ = the primes dividing $D$? William states that it's true.)

An immediate consequence is
\begin{Lemma}
If $f(x) = H_D(x)$, then $d_K \mid disc(L_f)$.
\end{Lemma}

This would allow us to search over a finite list of fields. Then an algorithm is as follows:

Find all possible $d_K$. For each $d_K$, compute ring class numbers, until the class numbers gets larger than $\deg(f)$. 



\end{document}
